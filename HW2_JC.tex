\documentclass[11pt,letterpaper]{article}
\usepackage[top=1in,textheight=9in]{geometry}
\usepackage{amsmath, amsthm, amssymb}
\usepackage{enumerate}
\usepackage{xfrac}

% Everything after a % sign is commented out.
% This is sometimes useful to write notes to yourself
% or to add spacing in the tex file so that it is easier
% to read.


% Some useful `macros'
% % % Feel free to define your own!
\newcommand{\C}{\mathbb{C}}
\newcommand{\N}{\mathbb{N}}
\newcommand{\R}{\mathbb{R}}
\newcommand{\Z}{\mathbb{Z}}
\newcommand{\eps}{\varepsilon}
\renewcommand{\epsilon}{\eps}


\newcommand{\prel}{{\rm prel}}


% Here is a pretty way to write down the problem
\newtheorem{innerprob}{Problem}
\newenvironment{prob}[1]
  {\renewcommand\theinnerprob{#1}\innerprob}
  {\endinnerprob}
% Here is a pretty way to wrote down the solution
\newenvironment{solution}
  {\renewcommand\qedsymbol{}\begin{proof}[Solution]}
  {\end{proof}\bigskip}



\setlength\parindent{0cm}
\setlength\parskip{5pt plus 1pt minus 1pt}



\title{Assignment \#2\\Math 584A}
\author{
	% PUT YOUR NAME HERE
	}
\date{Due September 17th, 1 am (via Gradescope)}









\begin{document}

\maketitle

For uploading to Gradescope, it will be easiest to put each solution on a different page.  The code for this is commented out in the tex file.

% Don't write anything between \begin{document}
% and \maketitle or it will show up before your name
% and the rest of the title stuff.



%State the problem
%\begin{prob}{PROBLEM #}
%  WRITE PROBLEM
%\end{prob}

% Note prob is custom-made above will



\begin{prob}{1}  % `prob' starts the (custom made, above) problem
			%  and the 
Let $(X,d_X)$ and $(Y,d_Y)$ be metric spaces that are both subsets of some larger linear space $Z$; that is, $X,Y\subset Z$.  Consider the following potential metrics on $X\cap Y$:
\begin{enumerate}[(i)]
	\item $d_p(v_1,v_2) = \left( d_X(v_1,v_2)^p + d_Y(v_1,v_2)^p\right)^\frac{1}{p}$ for a fixed $p\in [1,\infty)$,
	\item $d_{\max}(v_1,v_2) = \max\left\{ d_X(v_1,v_2), d_Y(v_1,v_2)\right\}$, and
	\item $d_{\min}(v_1,v_2) = \min\left\{ d_X(v_1,v_2), d_Y(v_1,v_2)\right\}$.
\end{enumerate}
Each of these is well-defined in that it is finite for any $v_1,v_2\in X\cap Y$.  Which are metrics?  Give a proof or counterexample for each.
\end{prob}
%Uncomment the lines below to solve the problem
\begin{solution}
	What follows is a proof of the metric in part (i). By construction, $d_p(v_1,v_2)$ can only be zero if both $d_X(v_1,v_2)$ and $d_Y(v_1,v_2)$ are zero, and will be positive if both $d_X(v_1,v_2)$ and $d_Y(v_1,v_2)$ are positive. Since both $d_X$ and $d_Y$ are metrics, they satisfy positive definiteness property. Thus, $d_p$ also satisfies positive definiteness.
	
	Since both $d_X$ and $d_Y$ are metrics, $$d_X(v_1,v_2) = d_X(v_2,v_1) \quad \text{ and } \quad d_Y(v_1,v_2) = d_Y(v_2,v_1).$$ Therefore, 
	\[\begin{split}
		d_p(v_2,v_1) &= \left( d_X(v_2,v_1)^p + d_Y(v_2,v_1)^p\right)^\frac{1}{p}\\
		&= \left( d_X(v_1,v_2)^p + d_Y(v_1,v_2)^p\right)^\frac{1}{p}\\
		&= d_p(v_1,v_2).
	\end{split}\]
	Thus, $d_p$ satisfies the symmetry property.
	
	
	Using the triangle inequality on the metrics $d_X$ and $d_Y$ in the definition for $d_p$ yields 
	\begin{equation}\label{e1}
	d_p(v_1,v_2) \leq \left ( [d_X(v_1,v_3) + d_X(v_3,v_2)]^p + [d_Y(v_1,v_3) + d_Y(v_3,v_2)]^p \right )^{\sfrac{1}{p}}.
	\end{equation}
	Let vectors $\bar x$ and $\bar y$ be two vectors defined by $$\bar x = (d_X(v_1,v_3), d_Y(v_1,v_3)) \quad \text{ and } \quad \bar y = (d_X(v_3,v_2), d_Y(v_3,v_2)).$$ Applying our definition for $|x|_p$ gives
	\[\begin{split}
		|\bar x +\bar y|_p = \left(|x_1 + y_1|^p + |x_2+y_2|^p\right)^{\sfrac{1}{p}}.
	\end{split}\]
	Substituting in the values for each vector gives
	\[\begin{split}
		|\bar x +\bar y|_p = \left(|d_X(v_1,v_3) + d_X(v_3,v_2)|^p + |d_Y(v_1,v_3)+d_Y(v_3,v_2)|^p\right)^{\sfrac{1}{p}}.
	\end{split}\]
	The right hand side is half of the inequality in \eqref{e1}, and thus 
	\begin{equation}\label{e2}
	d_p(v_1, v_2) \leq |\bar x +\bar y|_p.
	\end{equation}
	Since we proved in a previous homework that $|\cdot|_p$ satisfies the triangle inequality,
	\[\begin{split}
		|\bar x +\bar y|_p \leq |\bar x|_p + |\bar y|_p.
	\end{split}\]
	Substituting in the values for each vector gives 
	\[\begin{split}
		|\bar x +\bar y|_p &\leq |d_X(v_1,v_3) + d_Y(v_1,v_3)|_p + |d_X(v_3,v_2)+ d_Y(v_3,v_2)|_p\\
		& = d_p(v_1,v_3)^p + d_p(v_3,v_2)^p
	\end{split}\]
	Combining with \eqref{e2} gives $$d_p(v_1, v_2) \leq d_p(v_1,v_3)^p + d_p(v_3,v_2)^p,$$ thus satisfying the triangle inequality. Therefore, $d_p$ is a metric.
	
	What follows is a proof of the metric in part (ii). Since $d_X$ and $d_Y$ are metrics $$\max\left\{ d_X(v_1,v_2), d_Y(v_1,v_2)\right\} \geq 0.$$ And since $d_X(v_1,v_2), d_Y(v_1,v_2) = 0$ if and only if $v_1 = v_2$, the same must be true for $d_{max}(v_1,v_2)$. Thus, $d_{max}$ satisfies positive definiteness.
	
	Consider $d_{\max}(v_2,v_1)$. But the symmetry property of $d_X$ and $d_Y$
	\[\begin{split}
		d_{\max}(v_2,v_1) &= \max\left\{ d_X(v_2,v_1), d_Y(v_2,v_1)\right\}\\
		&=\max\left\{ d_X(v_1,v_2), d_Y(v_1,v_2)\right\}\\
		&= d_{\max}(v_1,v_2).
	\end{split}\] Therefore, $d_{\max}$ satisfies the symmetry property.
	
	Using the triangle property on $d_X$ and $d_Y$ gives $$d_{\max}(v_1,v_2) \leq \max\left\{\left ( d_X(v_1,v_3) + d_X(v_3,v_2) \right ), \left ( d_Y(v_1,v_3) + d_Y(v_3,v_2)\right ) \right \}.$$ Notice that 
	\[\begin{split}
		d_X(v_1,v_3) + d_X(v_3,v_2) &\leq \max\{d_X(v_1,v_3), d_Y(v_1,v_3)\} + d_X(v_3,v_2)\\
		&\leq \max\{d_X(v_1,v_3), d_Y(v_1,v_3)\} + \max\{d_X(v_3,v_2), d_Y(v_3,v_2)\}\\
		&= d_{\max}(v_1,v_3) + d_{\max}(v_3,v_2)
	\end{split}\] The same is also true for the sum $d_Y(v_1,v_3) + d_Y(v_3,v_2)$. Thus, 
	$$ \max\left\{\left ( d_X(v_1,v_3) + d_X(v_3,v_2) \right ), \left ( d_Y(v_1,v_3) + d_Y(v_3,v_2)\right ) \right \} \leq d_{\max}(v_1,v_3) + d_{\max}(v_3,v_2),$$ and $$
	d_{\max}(v_1,v_2) \leq d_{\max}(v_1,v_3) + d_{\max}(v_3,v_2).$$ Therefore, $d_{\max}$ satisfies the triangle property and is a metric. 
	
	What follows is a counterexample to the proposed metric in part (iii). Let $v_1 = (1,1)$ and $v_2 = $
	 
\end{solution}
\newpage



%
%
%\begin{prob}{2}  % `prob' starts the (custom made, above) problem
%			%  and the 
%Is there a constant $C>0$ such that, for any $f \in L^1(\R_+) \cap L^7(\R_+)$,
%\[
%	\|f\|_{L^1\cap L^7} \leq C \|f\|_{L^2}?
%\]
%If the answer is yes, prove it.  If the answer is no, establish this by finding a sequence $f_n \in L^1\cap L^7$ such that $\|f_n\|_{L^1 \cap L^7} \geq n \|f_n\|_2$ for each $n$.
%\end{prob}
%%Uncomment the lines below to solve the problem
%%\begin{solution}
%%This is a very elegant solution.
%%\end{solution}
%%\newpage



\begin{prob}{2}  % `prob' starts the (custom made, above) problem
			%  and the 
Fix a metric space $(X,d)$.  Show that if $U_1, U_2, \dots$ are open subsets of $X$ then so is
\[
	\bigcup_{i=1}^\infty U_i = \{ u \in X : \text{ there exists } i\in\N \text{ such that } u \in U_i\}.
\]
\end{prob}
%Uncomment the lines below to solve the problem
\begin{solution}
	Fix any $$ x \in \bigcup_{i=1}^\infty U_i.$$ By the definition of union, there exists some $i_x$ such that $x\in U_{i_x}$. Since each $U_i$ is open, there exists $B_{r_x}(x)$ where $B_{r_x}(x) \subset U_{i_x}$. Because $$U_{i_x} \subset \bigcup_{i=1}^\infty U_i,$$ it must be true that $$B_{r_x}(x) \subset \bigcup_{i=1}^\infty U_i.$$ Since the choice of $x$ was arbitrary, the union of an arbitrary number of open sets is open. 
\end{solution}
\newpage




\begin{prob}{3}  % `prob' starts the (custom made, above) problem
			%  and the 
Let $(X, d)$ be a metric space where $X$ is nonempty and $d$ is the discrete metric.
\begin{enumerate}[(i)]
	
	\item Classify all continuous functions $f: \R \to X$ (you can take $\R$ to have the Euclidean metric).
	
	\item Classify all continuous functions $f: X \to \R$.
	
\end{enumerate}
\end{prob}
%Uncomment the lines below to solve the problem
\begin{solution}
	Let $f: \R \to X$ be continuous and let $\epsilon = 1$. Fix $x_0 \in \R$. By definition there exists some $\delta > 0$ such that for any $x\in \R$ that satisfies $|x-x_0| < \delta$, $d_{disc}(f(x),f(x_0)) < 1$. This must mean that $d_{disc}(f(x),f(x_0)) = 0$, which requires that $f(x) = f(x_0)$. Therefore, for $f$ to be continuous at $x_0$, it must be constant in the region around $f(x_0)$. Since the choice of $x_0$ was arbitrary, $f$ must be the constant function.
	
	Let $f: X \to \R$ be continuous. Fix $x_0 \in \R$ and consider when $|f(x) - f(x_0)| < \epsilon$ for some $\epsilon > 0$. By the definition of continuity, there must be some $\delta > 0$ such that $|f(x) - f(x_0)| < \epsilon$ whenever $d_{disc}(x,x_0) < \delta$. This is clearly true for the case when $x = x_0$, but when $x\not=x_0$, $d_{disc}(x,x_0) = 1$. Therefore, whenever $|f(x) - f(x_0)| < \epsilon$ and $x \not=x_0$,  $d_{disc}(x,x_0) = 1$. Thus $\delta > 1$ in order to satisfy the definition of continuity. But since $d_{disc}(x,x_0) = 1$ for all $x \not= x_0$, $|f(x) - f(x_0)| < \epsilon$ for all $x \in X$. Since the choice of $x_0$ was arbitrary, every function is continuous everywhere. 
	
\end{solution}
\newpage



\begin{prob}{4} % `prob' starts the (custom made, above) problem
			%  and the 
Fix any non-negative function $K \in L_\prel^1(\R)$.  Define a function
\[
	T: L_\prel^1([0,1]) \to L_\prel^1([0,1])
\]
by, for every $f \in L_\prel^1([0,1])$,
\[
	(Tf)(x) = \int_0^1 K(x-y) f(y) dy.
\]
\begin{enumerate}[(i)]
	\item Show that $T$ is well-defined (i.e. $Tf \in L_\prel^1([0,1])$ whenever $f\in L_{\rm prel}^1([0,1])$).  Note: you can exchange the order of integration freely in this problem.  We will justify this later in the course.
	
	Also, you may find it helpful to show that if $f\in L^1_{\rm prel}$, there are nonnegative $f_+, f_- \in L^1_{\rm prel}$ such that $f = f_+ - f_-$.
	
	\item Show that $T$ is continuous.  
	\item Is $T$ uniformly continuous?
\end{enumerate}
\end{prob}
%Uncomment the lines below to solve the problem
\begin{solution}
 	We begin proving $Tf\in L_\prel^1([0,1])$ by first showing that $Tf$ is bounded. Let $f\in L_{\rm prel}^1([0,1])$. By definition of the $L_\prel^1([0,1])$ norm, 
 	\[\begin{split}
 		||(Tf)(x)||_{L_\prel^1([0,1])} = \int_{0}^{1}\left |\int_{0}^{1}K(x-y)f(y)dy\right |dx
 	\end{split}\]
 	We proceed by showing that for any $f\in L_\prel^1(\R)$, $$ \left |\int_{-\infty}^{\infty}f(x)dx \right | \leq \int_{-\infty}^{\infty}|f(x)|dx.$$ Let $f = f_+ + f_-$ where $$f_+(x) = \{f(x):f(x) \geq 0\} \quad \text{and} \quad f_-(x) = \{f(x):f(x) < 0\}.$$ Notice that $f = f_+ - (-f_-)$. Taking the absolute value of the integral of both sides gives $$\left |\int_{-\infty}^{\infty} f(x)dx \right | = \left |\int_{-\infty}^{\infty}f_+(x)dx - \int_{-\infty}^{\infty}(-f_-(x))dx \right |.$$ Since, $f \in  L_\prel^1(\R)$, these integrals must be finite and the triangle inequality gives $$\left |\int_{-\infty}^{\infty} f(x)dx \right | \leq \left |\int_{-\infty}^{\infty}f_+(x)dx \right | + \left | \int_{-\infty}^{\infty}(-f_-(x))dx \right |.$$ Since both $f_+(x)$ and $-f_-(x)$ are necessarily positive, the absolute value signs can be dropped giving 
 	\[\begin{split}
 		\left |\int_{-\infty}^{\infty} f(x)dx \right | &\leq \int_{-\infty}^{\infty}f_+(x)dx + \int_{-\infty}^{\infty}(-f_-(x))dx\\
 		&= \int_{-\infty}^{\infty}(f_+(x)-f_-(x))dx
 	\end{split}\]
 	Notice that 
 	\[f_+(x)-f_-(x) =
 	\begin{cases}
 		f(x) \quad &\text{ if } f(x)\geq 0\\
 		-f(x) \quad &\text{ if } f(x) <0,
 	\end{cases} \]
 	which is the same as the definition of $|f(x)|$. Thus, $$\left |\int_{-\infty}^{\infty} f(x)dx \right | \leq \int_{-\infty}^{\infty}|f(x)|dx.$$ Continuing with the evaluation of $||(Tf)(x)||_{L_\prel^1([0,1])}$ gives
 	\[\begin{split}
 		||(Tf)(x)||_{L_\prel^1([0,1])} &\leq  \int_{0}^{1} \int_{0}^{1} |K(x-y)f(y) |dydx\\
 		&\leq \int_{0}^{1} \int_{0}^{1} |K(x-y)||f(y)|dydx\\
 		&=\int_{0}^{1} |f(y)|\int_{0}^{1} |K(x-y)|dxdy\\
 		&\leq \int_{0}^{1} |f(y)|\int_{-\infty}^{\infty} |K(x-y)|dxdy.
 	\end{split}\] 
 	Since $K \in L_\prel^1(\R)$, there exists some $K \in R$ such that $$\int_{-\infty}^{\infty}|K(x)|dx \leq K.$$ Thus,
 	\[\begin{split}
 		\int_{0}^{1} |f(y)|\int_{-\infty}^{\infty} |K(x-y)|dxdy \leq K \int_{0}^{1} |f(y)|dy.
 	\end{split}\] And since, $f \in L_\prel^1([0,1])$, there exists some $F \in R$ such that $$\int_{0}^{1} |f(y)|dy \leq F.$$ Thereby, $$||(Tf)(x)||_{L_\prel^1([0,1])} \leq KF,$$ and $Tf$ must be bounded. 
 	
 	To show that $Tf$ is continuous, fix any $\epsilon > 0$ and $x_0 \in \R$. For some other $x\in \R$,
 	\[\begin{split}
 		|Tf(x)-Tf(x_0)| &= \left | \int_{0}^{1} K(x-y)f(y)dy - \int_{0}^{1} K(x_0-y)f(y)dy \right |\\
 		& = \left | \int_{0}^{1} \left (K(x-y) - K(x_0-y) \right ) f(y) dy \right |\\
 		&\leq \int_{0}^{1} \left | \left (K(x-y) - K(x_0-y) \right ) \right | |f(y)| dy.
 	\end{split}\]
 	Since $K \in L_\prel^1(\R)$, $K$ must be continuous. Thus for $F$ as defined earlier, there must exist some $\delta > 0$ such that $|x-x_0| < \delta$ implies $|K(x) - K(x_0)| < \sfrac{\epsilon}{F}$. Notice that
 	\[\begin{split}
 		|x-x_0| &= |x - y + y - x_0|\\
 		&= |(x - y) - (x_0-y)|,
 	\end{split}\] and let $|x-x_0| < \delta$. Then,
 	\[\begin{split}
 		|Tf(x)-Tf(x_0)| &\leq \frac{\epsilon}{F} \int_{0}^{1}  |f(y)| dy\\
 		&\leq \frac{\epsilon}{F} (F)\\
 		& = \epsilon.
 	\end{split}\]
 	Because the choice of $f$ was arbitrary, $Tf$ is continuous. Consequently, $Tf \in L_\prel^1([0,1])$.
 	
 	To show that $T$ is continuous, fix any $\epsilon > 0$, and let $f\in L_{\rm prel}^1([0,1])$. Let $g\in L_{\rm prel}^1([0,1])$ such that $||f-g||_{L_\prel^1([0,1])} < \sfrac{\epsilon}{K}$ where $K$ is the same value defined above. Then,
 	\[\begin{split}
 		||Tf(x)-Tg(x)||_{L_\prel^1([0,1])} &= \int_{0}^{1} \left | \int_{0}^{1} K(x-y)f(y)dy - \int_{0}^{1} K(x-y)g(y)dy \right | dx \\
 		&= \int_{0}^{1} \left | \int_{0}^{1} K(x-y)(f(y) - g(y) ) dy \right | dx \\
 		&\leq \int_{0}^{1} \int_{0}^{1} |K(x-y)||(f(y) - g(y))| dy dx.
 	\end{split}\]
 	Using the same value for $K$ defined above,
 	\[\begin{split}
 		||Tf(x)-Tg(x)||_{L_\prel^1([0,1])} &\leq K \int_{0}^{1}|(f(y) - g(y))| dy\\
 		&= K ||f-g||_{L_\prel^1([0,1])}.
	\end{split}\]
	But by assumption,
	\[\begin{split}
		||Tf(x)-Tg(x)||_{L_\prel^1([0,1])} &\leq K \frac{\epsilon}{K}\\
		&= \epsilon.
	\end{split}\]
	Since the choice of $f$ was arbitrary, $T$ is continuous.
	
	Since the value of $\delta$ only depends on $K$, $T$ and not on the choice of $f$ or $g$, $T$ is uniformly continuous.
\end{solution}
\newpage




%
%\begin{prob}{5} % `prob' starts the (custom made, above) problem
%			%  and the 
%Show that H\"older's inequality is sharp in the following sense: given $f \in L^1([a,b])$ and $\eps>0$, there is $g_\eps \in L^\infty([a,b])$ such that
%\[\tag{$*$}
%	\int_a^b f(x) g_\eps(x) dx
%		\geq \|g_\eps\|_\infty (\|f\|_1  - \eps).
%\]
%Similarly, show that for every\footnote{You may assume that $g \in C([a,b])$.} $g \in L^\infty([a,b])$ and $\eps>0$, there is $f_\eps \in L^1([a,b])$ such that
%\[\tag{$**$}
%	\int_a^b f_\eps(x) g(x) dx
%		\geq \|f_\eps\|_1 (\|g\|_\infty - \eps).
%\]
%Do we need the $\eps$ in both ($*$) and ($**$)?  That is, given any $f\in L^1$ is there a $g_{\eps=0}\in L^\infty$ so that ($*$) holds with $\eps=0$ and, given any $g\in L^\infty$, is there a $f_{\eps=0} \in L^1$ such that ($**$) holds with $\eps=0$?
%\end{prob}
%%Uncomment the lines below to solve the problem
%%\begin{solution}
%%This is a very elegant solution.
%%\end{solution}
%%\newpage





\begin{prob}{5} % `prob' starts the (custom made, above) problem
			%  and the 
Fix a metric space $(X,d)$.  Suppose that $x_0 \in X$ and $r>0$.  Show that $B_r(x_0)$ is an open set.
\end{prob}
%Uncomment the lines below to solve the problem
\begin{solution}
	The first case is if $B_r(x_0)$ is empty. But the empty set is open.
	
	The second case is when $B_r(x_0)$ is nonempty. Let $y\in B_r(x_0)$. Thus, $d(x_0,y) < r$. There must then exist $s \in \R$ such that $$d(x_0,y) + s < r.$$ Let $B_s(y)$ be the open ball centered at $y$ with radius $s$. Notice that $B_s(y)$ cannot be empty as $y\in B_s(y)$ by definition. Let $z \in B_s(y)$ be any element of $B_s(y)$. Thus, $d(y,z) < s$. Therefore, $d(x_0,y) + d(y,z) < r$, and by the triangle inequality $$d(x_0,z) \leq d(x_0,y) + d(y,z) < r.$$ Therefore, $z\in B_r(x_0)$. Since the choice of $z$ was arbitrary, $B_r(x_0)$ is open. 
	
	 
\end{solution}
\newpage





\begin{prob}{6} % `prob' starts the (custom made, above) problem
			%  and the 
Fix any constant $\theta \in \R$ and let $x_n$ be a convergent sequence.  Show that
\[
	\lim_{n\to\infty} \theta x_n = \theta \lim_{n\to\infty}x_n.
\]
\end{prob}
%Uncomment the lines below to solve the problem
\begin{solution}
	Since the sequence $(x_n)$ converges, let $$ \lim_{n\to\infty} x_n = x.$$ Then for any $x_n$, consider $|\theta x_n - \theta x| = |\theta||x_n - x|$. Consider the case when $\theta \not= 0$. Since, $(x_n)$ converges to $x$, there exists some $N \in \N$ such that whenever $n \geq N$, $$|x_n - x| < \frac{\epsilon}{|\theta |}.$$ Thus, for $n\geq N$, 
	\[\begin{split}
		|\theta x_n - \theta x| &< |\theta|\frac{\epsilon}{|\theta |}\\
		&= \epsilon.
	\end{split}\]
	Therefore, $\theta x$ is the limit of the sequence $(\theta x_n)$.
	
	If $\theta = 0$, then each term in the sequence $(\theta x_n)$ is $0$, and sequence thus converges to $0$. Because $(x_n)$ converges $$\theta \lim_{n\to\infty}x_n = 0.$$ Since both terms are $0$, the equality holds. 
\end{solution}
\newpage




\begin{prob}{7} % `prob' starts the (custom made, above) problem
			%  and the 
Suppose that $(X,d)$ is a metric space and $f, g: (X,d) \to \R$ are continuous functions.
\begin{enumerate}[(i)]
	\item Show that $f+g$ is continuous.
	
	\item Show that $f\cdot g$ is continuous.  
	
	\item If $f,g$ are uniformly continuous, show that $f+g$ is uniformly continuous.
	
	\item If $f,g$ are uniformly continuous, is $f\cdot g$ uniformly continuous?
	
\end{enumerate}
\end{prob}
%Uncomment the lines below to solve the problem
\begin{solution}
	To show that $f+g$ is continuous, fix some $\epsilon > 0$ and some $x_0 \in X$. For some other $x\in X$, 
	\[\begin{split}
		|(f(x) + g(x)) - (f(x_0) + g(x_0))| &=  | (f(x) - f(x_0)) + (g(x) - g(x_0))|\\
		& \leq | f(x) - f(x_0) |+|g(x) - g(x_0)|.
	\end{split}\]
	Since both $f$ and $g$ are continuous, there exists $\delta_f$ and $\delta_g$ such that $|f(x) - f(x_0)| < \sfrac{\epsilon}{2}$ whenever $d(x,x_0) < \delta_f$ and $|g(x) - g(x_0)| < \sfrac{\epsilon}{2}$ whenever $d(x,x_0) < \delta_g$. Let $\delta = \max\{\delta_f, \delta_g\}$. Then, $d(x,x_0) < \delta$ implies 
	\[\begin{split}
		|(f(x) + g(x)) - (f(x_0) + g(x_0))| & \leq | f(x) - f(x_0) |+|g(x) - g(x_0)|\\
		&<\frac{\epsilon}{2} + \frac{\epsilon}{2}\\
		&= \epsilon.
	\end{split}\]
	Since the choice of $x_0$ was arbitrary, $f+g$ is continuous.
	
	For part (ii), fix some $\epsilon > 0$ and some $x_0 \in X$. For some other $x\in X$, 
	\[\begin{split}
		|f(x)g(x) - f(x_0)g(x_0)| &=  |f(x)g(x) - f(x_0)g(x) + f(x_0)g(x) + f(x_0)g(x_0)|\\
		&= |g(x)(f(x) - f(x_0)) + f(x_0)(g(x) - g(x_0))|\\
		&\leq |g(x)(f(x) - f(x_0))| + |f(x_0)(g(x) - g(x_0))|\\
		&\leq |g(x)||f(x) - f(x_0)| + |f(x_0)||g(x) - g(x_0)|.
	\end{split}\]
	Since both $f$ and $g$ are continuous, there exists $\delta_f$ and $\delta_g$ such that $|f(x) - f(x_0)| < \epsilon_f$ whenever $d(x,x_0) < \delta_f$, and $|g(x) - g(x_0)| < \epsilon_f$ whenever $d(x,x_0) < \delta_g$. Let $\delta = \max\{\delta_f, \delta_g\}$ and let $d(x,x_0) < \delta$. To get an upper bound for $|g(x)|$, let $A = (x_0 - \delta, x_0+\delta)$ be an interval and let $$G = \sup_{x \in A}|g(x)|.$$ Let $$\epsilon_f = \frac{\epsilon}{2G} \quad \text{and} \quad \epsilon_g = \frac{\epsilon}{2|f(x_0)|}.$$ Then $d(x,x_0) < \delta$ implies,
	\[\begin{split}
		|f(x)g(x) - f(x_0)g(x_0)| & \leq G\frac{\epsilon}{2G} + |f(x_0)|\frac{\epsilon}{2|f(x_0)|}\\
		& = \frac{\epsilon}{2} + \frac{\epsilon}{2}\\
		&= \epsilon.
	\end{split}\]
	Since the choice of $x_0$ was arbitrary, $f\cdot g$ is continuous.
	
	For part (iii), fix any $\epsilon > 0$ and any $x,y \in X$. Then, 
	\[\begin{split}
		|(f(x) + g(x)) - (f(y) + g(y))| &=  | (f(x) - f(y)) + (g(x) - g(y))|\\
		& \leq | f(x) - f(y) |+|g(x) - g(y)|.
	\end{split}\]
	Since both $f$ and $g$ are uniformly continuous, there exists $\delta_f$ and $\delta_g$ such that $|f(x) - f(y)| < \sfrac{\epsilon}{2}$ whenever $d(x,y) < \delta_f$ and $|g(x) - g(y)| < \sfrac{\epsilon}{2}$ whenever $d(x,y) < \delta_g$. Let $\delta = \max\{\delta_f, \delta_g\}$. Then, $d(x,x_0) < \delta$ implies 
	\[\begin{split}
		|(f(x) + g(x)) - (f(x_0) + g(x_0))| & \leq | f(x) - f(x_0) |+|g(x) - g(x_0)|\\
		&<\frac{\epsilon}{2} + \frac{\epsilon}{2}\\
		&= \epsilon.
	\end{split}\]
	Since the choice of $x$ and $y$ was arbitrary, and in neither case does the choice of $\delta$ depend on $x$ or $y$, $f+g$ is uniformly continuous.
	
	To show that $f\cdot g$ is not necessarily uniformly continuous, let $f,g = x$. We demonstrated in class that this function is uniformly continuous. Assume for contradiction that $f\cdot g = x^2$ is uniformly continuous. Then there exists $\delta$ such that for any $x,y \in X$ where $d(x,y) < \delta$, $|x^2-y^2| < 1$. Let $y = x + \sfrac{\delta}{2}$. Then,
	\[\begin{split}
		1&> |x^2-y^2| \\
		&=|x^2-(x + \sfrac{\delta}{2})^2|\\
		&= |x^2-(x^2 + x\delta + \sfrac{\delta^2}{4})|\\
		&=|x\delta +\sfrac{\delta^2}{4}|.
	\end{split}\]
	Since $\delta$ is positive, this value is maximized when $x$ is positive. In this case, the absolute value can be dropped and rearranging for $x$ yields $$\frac{1-\sfrac{\delta^2}{4}}{\delta} > x.$$ This inequality sets an upper bound for $x$, and therefore does not apply for all $x$ and $y$ in $\R$. Therefore, $f\cdot g$ is not uniformly continuous, despite $f$ and $g$ both being uniformly continuous.
	
\end{solution}
%\newpage




\end{document}